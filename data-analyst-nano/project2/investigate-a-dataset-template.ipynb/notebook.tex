
% Default to the notebook output style

    


% Inherit from the specified cell style.




    
\documentclass[11pt]{article}

    
    
    \usepackage[T1]{fontenc}
    % Nicer default font (+ math font) than Computer Modern for most use cases
    \usepackage{mathpazo}

    % Basic figure setup, for now with no caption control since it's done
    % automatically by Pandoc (which extracts ![](path) syntax from Markdown).
    \usepackage{graphicx}
    % We will generate all images so they have a width \maxwidth. This means
    % that they will get their normal width if they fit onto the page, but
    % are scaled down if they would overflow the margins.
    \makeatletter
    \def\maxwidth{\ifdim\Gin@nat@width>\linewidth\linewidth
    \else\Gin@nat@width\fi}
    \makeatother
    \let\Oldincludegraphics\includegraphics
    % Set max figure width to be 80% of text width, for now hardcoded.
    \renewcommand{\includegraphics}[1]{\Oldincludegraphics[width=.8\maxwidth]{#1}}
    % Ensure that by default, figures have no caption (until we provide a
    % proper Figure object with a Caption API and a way to capture that
    % in the conversion process - todo).
    \usepackage{caption}
    \DeclareCaptionLabelFormat{nolabel}{}
    \captionsetup{labelformat=nolabel}

    \usepackage{adjustbox} % Used to constrain images to a maximum size 
    \usepackage{xcolor} % Allow colors to be defined
    \usepackage{enumerate} % Needed for markdown enumerations to work
    \usepackage{geometry} % Used to adjust the document margins
    \usepackage{amsmath} % Equations
    \usepackage{amssymb} % Equations
    \usepackage{textcomp} % defines textquotesingle
    % Hack from http://tex.stackexchange.com/a/47451/13684:
    \AtBeginDocument{%
        \def\PYZsq{\textquotesingle}% Upright quotes in Pygmentized code
    }
    \usepackage{upquote} % Upright quotes for verbatim code
    \usepackage{eurosym} % defines \euro
    \usepackage[mathletters]{ucs} % Extended unicode (utf-8) support
    \usepackage[utf8x]{inputenc} % Allow utf-8 characters in the tex document
    \usepackage{fancyvrb} % verbatim replacement that allows latex
    \usepackage{grffile} % extends the file name processing of package graphics 
                         % to support a larger range 
    % The hyperref package gives us a pdf with properly built
    % internal navigation ('pdf bookmarks' for the table of contents,
    % internal cross-reference links, web links for URLs, etc.)
    \usepackage{hyperref}
    \usepackage{longtable} % longtable support required by pandoc >1.10
    \usepackage{booktabs}  % table support for pandoc > 1.12.2
    \usepackage[inline]{enumitem} % IRkernel/repr support (it uses the enumerate* environment)
    \usepackage[normalem]{ulem} % ulem is needed to support strikethroughs (\sout)
                                % normalem makes italics be italics, not underlines
    

    
    
    % Colors for the hyperref package
    \definecolor{urlcolor}{rgb}{0,.145,.698}
    \definecolor{linkcolor}{rgb}{.71,0.21,0.01}
    \definecolor{citecolor}{rgb}{.12,.54,.11}

    % ANSI colors
    \definecolor{ansi-black}{HTML}{3E424D}
    \definecolor{ansi-black-intense}{HTML}{282C36}
    \definecolor{ansi-red}{HTML}{E75C58}
    \definecolor{ansi-red-intense}{HTML}{B22B31}
    \definecolor{ansi-green}{HTML}{00A250}
    \definecolor{ansi-green-intense}{HTML}{007427}
    \definecolor{ansi-yellow}{HTML}{DDB62B}
    \definecolor{ansi-yellow-intense}{HTML}{B27D12}
    \definecolor{ansi-blue}{HTML}{208FFB}
    \definecolor{ansi-blue-intense}{HTML}{0065CA}
    \definecolor{ansi-magenta}{HTML}{D160C4}
    \definecolor{ansi-magenta-intense}{HTML}{A03196}
    \definecolor{ansi-cyan}{HTML}{60C6C8}
    \definecolor{ansi-cyan-intense}{HTML}{258F8F}
    \definecolor{ansi-white}{HTML}{C5C1B4}
    \definecolor{ansi-white-intense}{HTML}{A1A6B2}

    % commands and environments needed by pandoc snippets
    % extracted from the output of `pandoc -s`
    \providecommand{\tightlist}{%
      \setlength{\itemsep}{0pt}\setlength{\parskip}{0pt}}
    \DefineVerbatimEnvironment{Highlighting}{Verbatim}{commandchars=\\\{\}}
    % Add ',fontsize=\small' for more characters per line
    \newenvironment{Shaded}{}{}
    \newcommand{\KeywordTok}[1]{\textcolor[rgb]{0.00,0.44,0.13}{\textbf{{#1}}}}
    \newcommand{\DataTypeTok}[1]{\textcolor[rgb]{0.56,0.13,0.00}{{#1}}}
    \newcommand{\DecValTok}[1]{\textcolor[rgb]{0.25,0.63,0.44}{{#1}}}
    \newcommand{\BaseNTok}[1]{\textcolor[rgb]{0.25,0.63,0.44}{{#1}}}
    \newcommand{\FloatTok}[1]{\textcolor[rgb]{0.25,0.63,0.44}{{#1}}}
    \newcommand{\CharTok}[1]{\textcolor[rgb]{0.25,0.44,0.63}{{#1}}}
    \newcommand{\StringTok}[1]{\textcolor[rgb]{0.25,0.44,0.63}{{#1}}}
    \newcommand{\CommentTok}[1]{\textcolor[rgb]{0.38,0.63,0.69}{\textit{{#1}}}}
    \newcommand{\OtherTok}[1]{\textcolor[rgb]{0.00,0.44,0.13}{{#1}}}
    \newcommand{\AlertTok}[1]{\textcolor[rgb]{1.00,0.00,0.00}{\textbf{{#1}}}}
    \newcommand{\FunctionTok}[1]{\textcolor[rgb]{0.02,0.16,0.49}{{#1}}}
    \newcommand{\RegionMarkerTok}[1]{{#1}}
    \newcommand{\ErrorTok}[1]{\textcolor[rgb]{1.00,0.00,0.00}{\textbf{{#1}}}}
    \newcommand{\NormalTok}[1]{{#1}}
    
    % Additional commands for more recent versions of Pandoc
    \newcommand{\ConstantTok}[1]{\textcolor[rgb]{0.53,0.00,0.00}{{#1}}}
    \newcommand{\SpecialCharTok}[1]{\textcolor[rgb]{0.25,0.44,0.63}{{#1}}}
    \newcommand{\VerbatimStringTok}[1]{\textcolor[rgb]{0.25,0.44,0.63}{{#1}}}
    \newcommand{\SpecialStringTok}[1]{\textcolor[rgb]{0.73,0.40,0.53}{{#1}}}
    \newcommand{\ImportTok}[1]{{#1}}
    \newcommand{\DocumentationTok}[1]{\textcolor[rgb]{0.73,0.13,0.13}{\textit{{#1}}}}
    \newcommand{\AnnotationTok}[1]{\textcolor[rgb]{0.38,0.63,0.69}{\textbf{\textit{{#1}}}}}
    \newcommand{\CommentVarTok}[1]{\textcolor[rgb]{0.38,0.63,0.69}{\textbf{\textit{{#1}}}}}
    \newcommand{\VariableTok}[1]{\textcolor[rgb]{0.10,0.09,0.49}{{#1}}}
    \newcommand{\ControlFlowTok}[1]{\textcolor[rgb]{0.00,0.44,0.13}{\textbf{{#1}}}}
    \newcommand{\OperatorTok}[1]{\textcolor[rgb]{0.40,0.40,0.40}{{#1}}}
    \newcommand{\BuiltInTok}[1]{{#1}}
    \newcommand{\ExtensionTok}[1]{{#1}}
    \newcommand{\PreprocessorTok}[1]{\textcolor[rgb]{0.74,0.48,0.00}{{#1}}}
    \newcommand{\AttributeTok}[1]{\textcolor[rgb]{0.49,0.56,0.16}{{#1}}}
    \newcommand{\InformationTok}[1]{\textcolor[rgb]{0.38,0.63,0.69}{\textbf{\textit{{#1}}}}}
    \newcommand{\WarningTok}[1]{\textcolor[rgb]{0.38,0.63,0.69}{\textbf{\textit{{#1}}}}}
    
    
    % Define a nice break command that doesn't care if a line doesn't already
    % exist.
    \def\br{\hspace*{\fill} \\* }
    % Math Jax compatability definitions
    \def\gt{>}
    \def\lt{<}
    % Document parameters
    \title{investigate-a-dataset-template}
    
    
    

    % Pygments definitions
    
\makeatletter
\def\PY@reset{\let\PY@it=\relax \let\PY@bf=\relax%
    \let\PY@ul=\relax \let\PY@tc=\relax%
    \let\PY@bc=\relax \let\PY@ff=\relax}
\def\PY@tok#1{\csname PY@tok@#1\endcsname}
\def\PY@toks#1+{\ifx\relax#1\empty\else%
    \PY@tok{#1}\expandafter\PY@toks\fi}
\def\PY@do#1{\PY@bc{\PY@tc{\PY@ul{%
    \PY@it{\PY@bf{\PY@ff{#1}}}}}}}
\def\PY#1#2{\PY@reset\PY@toks#1+\relax+\PY@do{#2}}

\expandafter\def\csname PY@tok@w\endcsname{\def\PY@tc##1{\textcolor[rgb]{0.73,0.73,0.73}{##1}}}
\expandafter\def\csname PY@tok@c\endcsname{\let\PY@it=\textit\def\PY@tc##1{\textcolor[rgb]{0.25,0.50,0.50}{##1}}}
\expandafter\def\csname PY@tok@cp\endcsname{\def\PY@tc##1{\textcolor[rgb]{0.74,0.48,0.00}{##1}}}
\expandafter\def\csname PY@tok@k\endcsname{\let\PY@bf=\textbf\def\PY@tc##1{\textcolor[rgb]{0.00,0.50,0.00}{##1}}}
\expandafter\def\csname PY@tok@kp\endcsname{\def\PY@tc##1{\textcolor[rgb]{0.00,0.50,0.00}{##1}}}
\expandafter\def\csname PY@tok@kt\endcsname{\def\PY@tc##1{\textcolor[rgb]{0.69,0.00,0.25}{##1}}}
\expandafter\def\csname PY@tok@o\endcsname{\def\PY@tc##1{\textcolor[rgb]{0.40,0.40,0.40}{##1}}}
\expandafter\def\csname PY@tok@ow\endcsname{\let\PY@bf=\textbf\def\PY@tc##1{\textcolor[rgb]{0.67,0.13,1.00}{##1}}}
\expandafter\def\csname PY@tok@nb\endcsname{\def\PY@tc##1{\textcolor[rgb]{0.00,0.50,0.00}{##1}}}
\expandafter\def\csname PY@tok@nf\endcsname{\def\PY@tc##1{\textcolor[rgb]{0.00,0.00,1.00}{##1}}}
\expandafter\def\csname PY@tok@nc\endcsname{\let\PY@bf=\textbf\def\PY@tc##1{\textcolor[rgb]{0.00,0.00,1.00}{##1}}}
\expandafter\def\csname PY@tok@nn\endcsname{\let\PY@bf=\textbf\def\PY@tc##1{\textcolor[rgb]{0.00,0.00,1.00}{##1}}}
\expandafter\def\csname PY@tok@ne\endcsname{\let\PY@bf=\textbf\def\PY@tc##1{\textcolor[rgb]{0.82,0.25,0.23}{##1}}}
\expandafter\def\csname PY@tok@nv\endcsname{\def\PY@tc##1{\textcolor[rgb]{0.10,0.09,0.49}{##1}}}
\expandafter\def\csname PY@tok@no\endcsname{\def\PY@tc##1{\textcolor[rgb]{0.53,0.00,0.00}{##1}}}
\expandafter\def\csname PY@tok@nl\endcsname{\def\PY@tc##1{\textcolor[rgb]{0.63,0.63,0.00}{##1}}}
\expandafter\def\csname PY@tok@ni\endcsname{\let\PY@bf=\textbf\def\PY@tc##1{\textcolor[rgb]{0.60,0.60,0.60}{##1}}}
\expandafter\def\csname PY@tok@na\endcsname{\def\PY@tc##1{\textcolor[rgb]{0.49,0.56,0.16}{##1}}}
\expandafter\def\csname PY@tok@nt\endcsname{\let\PY@bf=\textbf\def\PY@tc##1{\textcolor[rgb]{0.00,0.50,0.00}{##1}}}
\expandafter\def\csname PY@tok@nd\endcsname{\def\PY@tc##1{\textcolor[rgb]{0.67,0.13,1.00}{##1}}}
\expandafter\def\csname PY@tok@s\endcsname{\def\PY@tc##1{\textcolor[rgb]{0.73,0.13,0.13}{##1}}}
\expandafter\def\csname PY@tok@sd\endcsname{\let\PY@it=\textit\def\PY@tc##1{\textcolor[rgb]{0.73,0.13,0.13}{##1}}}
\expandafter\def\csname PY@tok@si\endcsname{\let\PY@bf=\textbf\def\PY@tc##1{\textcolor[rgb]{0.73,0.40,0.53}{##1}}}
\expandafter\def\csname PY@tok@se\endcsname{\let\PY@bf=\textbf\def\PY@tc##1{\textcolor[rgb]{0.73,0.40,0.13}{##1}}}
\expandafter\def\csname PY@tok@sr\endcsname{\def\PY@tc##1{\textcolor[rgb]{0.73,0.40,0.53}{##1}}}
\expandafter\def\csname PY@tok@ss\endcsname{\def\PY@tc##1{\textcolor[rgb]{0.10,0.09,0.49}{##1}}}
\expandafter\def\csname PY@tok@sx\endcsname{\def\PY@tc##1{\textcolor[rgb]{0.00,0.50,0.00}{##1}}}
\expandafter\def\csname PY@tok@m\endcsname{\def\PY@tc##1{\textcolor[rgb]{0.40,0.40,0.40}{##1}}}
\expandafter\def\csname PY@tok@gh\endcsname{\let\PY@bf=\textbf\def\PY@tc##1{\textcolor[rgb]{0.00,0.00,0.50}{##1}}}
\expandafter\def\csname PY@tok@gu\endcsname{\let\PY@bf=\textbf\def\PY@tc##1{\textcolor[rgb]{0.50,0.00,0.50}{##1}}}
\expandafter\def\csname PY@tok@gd\endcsname{\def\PY@tc##1{\textcolor[rgb]{0.63,0.00,0.00}{##1}}}
\expandafter\def\csname PY@tok@gi\endcsname{\def\PY@tc##1{\textcolor[rgb]{0.00,0.63,0.00}{##1}}}
\expandafter\def\csname PY@tok@gr\endcsname{\def\PY@tc##1{\textcolor[rgb]{1.00,0.00,0.00}{##1}}}
\expandafter\def\csname PY@tok@ge\endcsname{\let\PY@it=\textit}
\expandafter\def\csname PY@tok@gs\endcsname{\let\PY@bf=\textbf}
\expandafter\def\csname PY@tok@gp\endcsname{\let\PY@bf=\textbf\def\PY@tc##1{\textcolor[rgb]{0.00,0.00,0.50}{##1}}}
\expandafter\def\csname PY@tok@go\endcsname{\def\PY@tc##1{\textcolor[rgb]{0.53,0.53,0.53}{##1}}}
\expandafter\def\csname PY@tok@gt\endcsname{\def\PY@tc##1{\textcolor[rgb]{0.00,0.27,0.87}{##1}}}
\expandafter\def\csname PY@tok@err\endcsname{\def\PY@bc##1{\setlength{\fboxsep}{0pt}\fcolorbox[rgb]{1.00,0.00,0.00}{1,1,1}{\strut ##1}}}
\expandafter\def\csname PY@tok@kc\endcsname{\let\PY@bf=\textbf\def\PY@tc##1{\textcolor[rgb]{0.00,0.50,0.00}{##1}}}
\expandafter\def\csname PY@tok@kd\endcsname{\let\PY@bf=\textbf\def\PY@tc##1{\textcolor[rgb]{0.00,0.50,0.00}{##1}}}
\expandafter\def\csname PY@tok@kn\endcsname{\let\PY@bf=\textbf\def\PY@tc##1{\textcolor[rgb]{0.00,0.50,0.00}{##1}}}
\expandafter\def\csname PY@tok@kr\endcsname{\let\PY@bf=\textbf\def\PY@tc##1{\textcolor[rgb]{0.00,0.50,0.00}{##1}}}
\expandafter\def\csname PY@tok@bp\endcsname{\def\PY@tc##1{\textcolor[rgb]{0.00,0.50,0.00}{##1}}}
\expandafter\def\csname PY@tok@fm\endcsname{\def\PY@tc##1{\textcolor[rgb]{0.00,0.00,1.00}{##1}}}
\expandafter\def\csname PY@tok@vc\endcsname{\def\PY@tc##1{\textcolor[rgb]{0.10,0.09,0.49}{##1}}}
\expandafter\def\csname PY@tok@vg\endcsname{\def\PY@tc##1{\textcolor[rgb]{0.10,0.09,0.49}{##1}}}
\expandafter\def\csname PY@tok@vi\endcsname{\def\PY@tc##1{\textcolor[rgb]{0.10,0.09,0.49}{##1}}}
\expandafter\def\csname PY@tok@vm\endcsname{\def\PY@tc##1{\textcolor[rgb]{0.10,0.09,0.49}{##1}}}
\expandafter\def\csname PY@tok@sa\endcsname{\def\PY@tc##1{\textcolor[rgb]{0.73,0.13,0.13}{##1}}}
\expandafter\def\csname PY@tok@sb\endcsname{\def\PY@tc##1{\textcolor[rgb]{0.73,0.13,0.13}{##1}}}
\expandafter\def\csname PY@tok@sc\endcsname{\def\PY@tc##1{\textcolor[rgb]{0.73,0.13,0.13}{##1}}}
\expandafter\def\csname PY@tok@dl\endcsname{\def\PY@tc##1{\textcolor[rgb]{0.73,0.13,0.13}{##1}}}
\expandafter\def\csname PY@tok@s2\endcsname{\def\PY@tc##1{\textcolor[rgb]{0.73,0.13,0.13}{##1}}}
\expandafter\def\csname PY@tok@sh\endcsname{\def\PY@tc##1{\textcolor[rgb]{0.73,0.13,0.13}{##1}}}
\expandafter\def\csname PY@tok@s1\endcsname{\def\PY@tc##1{\textcolor[rgb]{0.73,0.13,0.13}{##1}}}
\expandafter\def\csname PY@tok@mb\endcsname{\def\PY@tc##1{\textcolor[rgb]{0.40,0.40,0.40}{##1}}}
\expandafter\def\csname PY@tok@mf\endcsname{\def\PY@tc##1{\textcolor[rgb]{0.40,0.40,0.40}{##1}}}
\expandafter\def\csname PY@tok@mh\endcsname{\def\PY@tc##1{\textcolor[rgb]{0.40,0.40,0.40}{##1}}}
\expandafter\def\csname PY@tok@mi\endcsname{\def\PY@tc##1{\textcolor[rgb]{0.40,0.40,0.40}{##1}}}
\expandafter\def\csname PY@tok@il\endcsname{\def\PY@tc##1{\textcolor[rgb]{0.40,0.40,0.40}{##1}}}
\expandafter\def\csname PY@tok@mo\endcsname{\def\PY@tc##1{\textcolor[rgb]{0.40,0.40,0.40}{##1}}}
\expandafter\def\csname PY@tok@ch\endcsname{\let\PY@it=\textit\def\PY@tc##1{\textcolor[rgb]{0.25,0.50,0.50}{##1}}}
\expandafter\def\csname PY@tok@cm\endcsname{\let\PY@it=\textit\def\PY@tc##1{\textcolor[rgb]{0.25,0.50,0.50}{##1}}}
\expandafter\def\csname PY@tok@cpf\endcsname{\let\PY@it=\textit\def\PY@tc##1{\textcolor[rgb]{0.25,0.50,0.50}{##1}}}
\expandafter\def\csname PY@tok@c1\endcsname{\let\PY@it=\textit\def\PY@tc##1{\textcolor[rgb]{0.25,0.50,0.50}{##1}}}
\expandafter\def\csname PY@tok@cs\endcsname{\let\PY@it=\textit\def\PY@tc##1{\textcolor[rgb]{0.25,0.50,0.50}{##1}}}

\def\PYZbs{\char`\\}
\def\PYZus{\char`\_}
\def\PYZob{\char`\{}
\def\PYZcb{\char`\}}
\def\PYZca{\char`\^}
\def\PYZam{\char`\&}
\def\PYZlt{\char`\<}
\def\PYZgt{\char`\>}
\def\PYZsh{\char`\#}
\def\PYZpc{\char`\%}
\def\PYZdl{\char`\$}
\def\PYZhy{\char`\-}
\def\PYZsq{\char`\'}
\def\PYZdq{\char`\"}
\def\PYZti{\char`\~}
% for compatibility with earlier versions
\def\PYZat{@}
\def\PYZlb{[}
\def\PYZrb{]}
\makeatother


    % Exact colors from NB
    \definecolor{incolor}{rgb}{0.0, 0.0, 0.5}
    \definecolor{outcolor}{rgb}{0.545, 0.0, 0.0}



    
    % Prevent overflowing lines due to hard-to-break entities
    \sloppy 
    % Setup hyperref package
    \hypersetup{
      breaklinks=true,  % so long urls are correctly broken across lines
      colorlinks=true,
      urlcolor=urlcolor,
      linkcolor=linkcolor,
      citecolor=citecolor,
      }
    % Slightly bigger margins than the latex defaults
    
    \geometry{verbose,tmargin=1in,bmargin=1in,lmargin=1in,rmargin=1in}
    
    

    \begin{document}
    
    
    \maketitle
    
    

    
    \section{Project: Investigate a Movie
Dataset}\label{project-investigate-a-movie-dataset}

\subsection{Table of Contents}\label{table-of-contents}

Introduction

Data Wrangling

Exploratory Data Analysis

Conclusions

     \#\# Introduction

The data set analyzed in this notebook is the TMDB Movies Database. This
data set originiated from Kaggle and provided by Udacity. There is
information on more than 5000 movies. The information used below is
popularity, revenue, bugdet, and runtime. The information chosen from
the data set is to dive into what metrics are good for figuring out how
to measure a movies' success.

    

    \begin{Verbatim}[commandchars=\\\{\}]
{\color{incolor}In [{\color{incolor}1}]:} \PY{k+kn}{import} \PY{n+nn}{pandas} \PY{k}{as} \PY{n+nn}{pd}
        \PY{k+kn}{import} \PY{n+nn}{numpy} \PY{k}{as} \PY{n+nn}{np}
        \PY{k+kn}{import} \PY{n+nn}{scipy}\PY{n+nn}{.}\PY{n+nn}{stats} \PY{k}{as} \PY{n+nn}{st}
        \PY{k+kn}{import} \PY{n+nn}{matplotlib}\PY{n+nn}{.}\PY{n+nn}{pyplot} \PY{k}{as} \PY{n+nn}{plt}
        \PY{k+kn}{import} \PY{n+nn}{os}
        \PY{k+kn}{from} \PY{n+nn}{scipy}\PY{n+nn}{.}\PY{n+nn}{stats} \PY{k}{import} \PY{n}{pearsonr}
        \PY{k+kn}{import} \PY{n+nn}{seaborn} \PY{k}{as} \PY{n+nn}{sns}
        \PY{k+kn}{import} \PY{n+nn}{statsmodels}\PY{n+nn}{.}\PY{n+nn}{api} \PY{k}{as} \PY{n+nn}{sm}
        \PY{k+kn}{from} \PY{n+nn}{statsmodels}\PY{n+nn}{.}\PY{n+nn}{formula}\PY{n+nn}{.}\PY{n+nn}{api} \PY{k}{import} \PY{n}{ols}
        \PY{o}{\PYZpc{}}\PY{k}{matplotlib} inline
\end{Verbatim}


    \begin{Verbatim}[commandchars=\\\{\}]
/home/aurora/anaconda3/lib/python3.6/site-packages/statsmodels/compat/pandas.py:56: FutureWarning: The pandas.core.datetools module is deprecated and will be removed in a future version. Please use the pandas.tseries module instead.
  from pandas.core import datetools

    \end{Verbatim}

    \begin{Verbatim}[commandchars=\\\{\}]
{\color{incolor}In [{\color{incolor}2}]:} \PY{n}{pd}\PY{o}{.}\PY{n}{set\PYZus{}option}\PY{p}{(}\PY{l+s+s1}{\PYZsq{}}\PY{l+s+s1}{display.float\PYZus{}format}\PY{l+s+s1}{\PYZsq{}}\PY{p}{,} \PY{k}{lambda} \PY{n}{x}\PY{p}{:} \PY{l+s+s1}{\PYZsq{}}\PY{l+s+si}{\PYZpc{}.3f}\PY{l+s+s1}{\PYZsq{}} \PY{o}{\PYZpc{}} \PY{n}{x}\PY{p}{)} \PY{c+c1}{\PYZsh{} surppresses scietific notation}
\end{Verbatim}


     \#\# Data Wrangling

\subsubsection{General Properties}\label{general-properties}

    \begin{Verbatim}[commandchars=\\\{\}]
{\color{incolor}In [{\color{incolor}3}]:} \PY{c+c1}{\PYZsh{} Load your data and print out a few lines. Perform operations to inspect data}
        \PY{c+c1}{\PYZsh{}   types and look for instances of missing or possibly errant data.}
        \PY{n}{data} \PY{o}{=} \PY{n}{pd}\PY{o}{.}\PY{n}{read\PYZus{}csv}\PY{p}{(}\PY{l+s+s1}{\PYZsq{}}\PY{l+s+s1}{../tmdb\PYZhy{}movies.csv}\PY{l+s+s1}{\PYZsq{}}\PY{p}{)}
\end{Verbatim}


    \begin{Verbatim}[commandchars=\\\{\}]
{\color{incolor}In [{\color{incolor}4}]:} \PY{c+c1}{\PYZsh{}This allows us to look inside of the dataset}
        \PY{n}{data}\PY{o}{.}\PY{n}{head}\PY{p}{(}\PY{p}{)}
\end{Verbatim}


\begin{Verbatim}[commandchars=\\\{\}]
{\color{outcolor}Out[{\color{outcolor}4}]:}        id    imdb\_id  popularity     budget     revenue  \textbackslash{}
        0  135397  tt0369610      32.986  150000000  1513528810   
        1   76341  tt1392190      28.420  150000000   378436354   
        2  262500  tt2908446      13.113  110000000   295238201   
        3  140607  tt2488496      11.173  200000000  2068178225   
        4  168259  tt2820852       9.335  190000000  1506249360   
        
                         original\_title  \textbackslash{}
        0                Jurassic World   
        1            Mad Max: Fury Road   
        2                     Insurgent   
        3  Star Wars: The Force Awakens   
        4                     Furious 7   
        
                                                        cast  \textbackslash{}
        0  Chris Pratt|Bryce Dallas Howard|Irrfan Khan|Vi{\ldots}   
        1  Tom Hardy|Charlize Theron|Hugh Keays-Byrne|Nic{\ldots}   
        2  Shailene Woodley|Theo James|Kate Winslet|Ansel{\ldots}   
        3  Harrison Ford|Mark Hamill|Carrie Fisher|Adam D{\ldots}   
        4  Vin Diesel|Paul Walker|Jason Statham|Michelle {\ldots}   
        
                                                    homepage          director  \textbackslash{}
        0                      http://www.jurassicworld.com/   Colin Trevorrow   
        1                        http://www.madmaxmovie.com/     George Miller   
        2     http://www.thedivergentseries.movie/\#insurgent  Robert Schwentke   
        3  http://www.starwars.com/films/star-wars-episod{\ldots}       J.J. Abrams   
        4                           http://www.furious7.com/         James Wan   
        
                                 tagline      {\ldots}        \textbackslash{}
        0              The park is open.      {\ldots}         
        1             What a Lovely Day.      {\ldots}         
        2     One Choice Can Destroy You      {\ldots}         
        3  Every generation has a story.      {\ldots}         
        4            Vengeance Hits Home      {\ldots}         
        
                                                    overview runtime  \textbackslash{}
        0  Twenty-two years after the events of Jurassic {\ldots}     124   
        1  An apocalyptic story set in the furthest reach{\ldots}     120   
        2  Beatrice Prior must confront her inner demons {\ldots}     119   
        3  Thirty years after defeating the Galactic Empi{\ldots}     136   
        4  Deckard Shaw seeks revenge against Dominic Tor{\ldots}     137   
        
                                              genres  \textbackslash{}
        0  Action|Adventure|Science Fiction|Thriller   
        1  Action|Adventure|Science Fiction|Thriller   
        2         Adventure|Science Fiction|Thriller   
        3   Action|Adventure|Science Fiction|Fantasy   
        4                      Action|Crime|Thriller   
        
                                        production\_companies release\_date vote\_count  \textbackslash{}
        0  Universal Studios|Amblin Entertainment|Legenda{\ldots}       6/9/15       5562   
        1  Village Roadshow Pictures|Kennedy Miller Produ{\ldots}      5/13/15       6185   
        2  Summit Entertainment|Mandeville Films|Red Wago{\ldots}      3/18/15       2480   
        3          Lucasfilm|Truenorth Productions|Bad Robot     12/15/15       5292   
        4  Universal Pictures|Original Film|Media Rights {\ldots}       4/1/15       2947   
        
           vote\_average  release\_year    budget\_adj    revenue\_adj  
        0         6.500          2015 137999939.280 1392445892.524  
        1         7.100          2015 137999939.280  348161292.489  
        2         6.300          2015 101199955.472  271619025.408  
        3         7.500          2015 183999919.040 1902723129.802  
        4         7.300          2015 174799923.088 1385748801.471  
        
        [5 rows x 21 columns]
\end{Verbatim}
            
    \begin{Verbatim}[commandchars=\\\{\}]
{\color{incolor}In [{\color{incolor}5}]:} \PY{n}{data}\PY{o}{.}\PY{n}{shape} \PY{c+c1}{\PYZsh{}returns how many rows and column of the dataset}
\end{Verbatim}


\begin{Verbatim}[commandchars=\\\{\}]
{\color{outcolor}Out[{\color{outcolor}5}]:} (10866, 21)
\end{Verbatim}
            
    \begin{Verbatim}[commandchars=\\\{\}]
{\color{incolor}In [{\color{incolor}6}]:} \PY{n}{data}\PY{o}{.}\PY{n}{describe}\PY{p}{(}\PY{p}{)} \PY{c+c1}{\PYZsh{}returns statistics about the numerical columns}
\end{Verbatim}


\begin{Verbatim}[commandchars=\\\{\}]
{\color{outcolor}Out[{\color{outcolor}6}]:}               id  popularity        budget        revenue   runtime  \textbackslash{}
        count  10866.000   10866.000     10866.000      10866.000 10866.000   
        mean   66064.177       0.646  14625701.094   39823319.793   102.071   
        std    92130.137       1.000  30913213.831  117003486.582    31.381   
        min        5.000       0.000         0.000          0.000     0.000   
        25\%    10596.250       0.208         0.000          0.000    90.000   
        50\%    20669.000       0.384         0.000          0.000    99.000   
        75\%    75610.000       0.714  15000000.000   24000000.000   111.000   
        max   417859.000      32.986 425000000.000 2781505847.000   900.000   
        
               vote\_count  vote\_average  release\_year    budget\_adj    revenue\_adj  
        count   10866.000     10866.000     10866.000     10866.000      10866.000  
        mean      217.390         5.975      2001.323  17551039.823   51364363.253  
        std       575.619         0.935        12.813  34306155.723  144632485.040  
        min        10.000         1.500      1960.000         0.000          0.000  
        25\%        17.000         5.400      1995.000         0.000          0.000  
        50\%        38.000         6.000      2006.000         0.000          0.000  
        75\%       145.750         6.600      2011.000  20853251.084   33697095.717  
        max      9767.000         9.200      2015.000 425000000.000 2827123750.412  
\end{Verbatim}
            
    \begin{quote}
\textbf{Tip}: Make sure that you keep your reader informed on the steps
that you are taking in your investigation. Follow every code cell, or
every set of related code cells, with a markdown cell to describe to the
reader what was found in the preceding cell(s). Try to make it so that
the reader can then understand what they will be seeing in the following
cell(s).
\end{quote}

\subsubsection{Data Cleaning (Replace this with more specific
notes!)}\label{data-cleaning-replace-this-with-more-specific-notes}

    \begin{Verbatim}[commandchars=\\\{\}]
{\color{incolor}In [{\color{incolor}7}]:} \PY{c+c1}{\PYZsh{} Putting all of the columns that will be used in a data frame}
        \PY{n}{movie\PYZus{}data} \PY{o}{=} \PY{n}{data}\PY{p}{[}\PY{p}{[}\PY{l+s+s1}{\PYZsq{}}\PY{l+s+s1}{popularity}\PY{l+s+s1}{\PYZsq{}}\PY{p}{,} \PY{l+s+s1}{\PYZsq{}}\PY{l+s+s1}{budget}\PY{l+s+s1}{\PYZsq{}}\PY{p}{,} \PY{l+s+s1}{\PYZsq{}}\PY{l+s+s1}{revenue}\PY{l+s+s1}{\PYZsq{}}\PY{p}{,}\PY{l+s+s1}{\PYZsq{}}\PY{l+s+s1}{runtime}\PY{l+s+s1}{\PYZsq{}}\PY{p}{,}\PY{l+s+s1}{\PYZsq{}}\PY{l+s+s1}{genres}\PY{l+s+s1}{\PYZsq{}}\PY{p}{]}\PY{p}{]}
\end{Verbatim}


    \begin{Verbatim}[commandchars=\\\{\}]
{\color{incolor}In [{\color{incolor}8}]:} \PY{c+c1}{\PYZsh{}checks the new data set}
        \PY{n}{movie\PYZus{}data}\PY{o}{.}\PY{n}{shape} 
\end{Verbatim}


\begin{Verbatim}[commandchars=\\\{\}]
{\color{outcolor}Out[{\color{outcolor}8}]:} (10866, 5)
\end{Verbatim}
            
    \begin{Verbatim}[commandchars=\\\{\}]
{\color{incolor}In [{\color{incolor}9}]:} \PY{c+c1}{\PYZsh{}Drops the rows that have atleast 1 NAN}
        \PY{n}{clean\PYZus{}mdata} \PY{o}{=} \PY{n}{movie\PYZus{}data}\PY{o}{.}\PY{n}{dropna}\PY{p}{(}\PY{p}{)} 
\end{Verbatim}


    \begin{Verbatim}[commandchars=\\\{\}]
{\color{incolor}In [{\color{incolor}10}]:} \PY{c+c1}{\PYZsh{}Now we take a peak at the clean dataset}
         \PY{n}{clean\PYZus{}mdata}\PY{o}{.}\PY{n}{describe}\PY{p}{(}\PY{p}{)}
\end{Verbatim}


\begin{Verbatim}[commandchars=\\\{\}]
{\color{outcolor}Out[{\color{outcolor}10}]:}        popularity        budget        revenue   runtime
         count   10843.000     10843.000      10843.000 10843.000
         mean        0.647  14656724.439   39907792.389   102.138
         std         1.001  30938637.671  117113132.251    31.293
         min         0.000         0.000          0.000     0.000
         25\%         0.208         0.000          0.000    90.000
         50\%         0.385         0.000          0.000    99.000
         75\%         0.715  15000000.000   24136754.000   111.000
         max        32.986 425000000.000 2781505847.000   900.000
\end{Verbatim}
            
    \begin{Verbatim}[commandchars=\\\{\}]
{\color{incolor}In [{\color{incolor}11}]:} \PY{c+c1}{\PYZsh{} drop if there are duplicates, just in case}
         \PY{n}{clean\PYZus{}mdata} \PY{o}{=} \PY{n}{clean\PYZus{}mdata}\PY{o}{.}\PY{n}{drop\PYZus{}duplicates}\PY{p}{(}\PY{p}{)}
\end{Verbatim}


    \begin{Verbatim}[commandchars=\\\{\}]
{\color{incolor}In [{\color{incolor}12}]:} \PY{n}{clean\PYZus{}mdata} \PY{o}{=} \PY{n}{clean\PYZus{}mdata}\PY{p}{[}\PY{p}{(}\PY{n}{clean\PYZus{}mdata} \PY{o}{!=} \PY{l+m+mi}{0}\PY{p}{)}\PY{o}{.}\PY{n}{all}\PY{p}{(}\PY{l+m+mi}{1}\PY{p}{)}\PY{p}{]}
\end{Verbatim}


    \begin{Verbatim}[commandchars=\\\{\}]
{\color{incolor}In [{\color{incolor}13}]:} \PY{c+c1}{\PYZsh{}take another peak}
         \PY{n}{clean\PYZus{}mdata}\PY{o}{.}\PY{n}{head}\PY{p}{(}\PY{p}{)}
\end{Verbatim}


\begin{Verbatim}[commandchars=\\\{\}]
{\color{outcolor}Out[{\color{outcolor}13}]:}    popularity     budget     revenue  runtime  \textbackslash{}
         0      32.986  150000000  1513528810      124   
         1      28.420  150000000   378436354      120   
         2      13.113  110000000   295238201      119   
         3      11.173  200000000  2068178225      136   
         4       9.335  190000000  1506249360      137   
         
                                               genres  
         0  Action|Adventure|Science Fiction|Thriller  
         1  Action|Adventure|Science Fiction|Thriller  
         2         Adventure|Science Fiction|Thriller  
         3   Action|Adventure|Science Fiction|Fantasy  
         4                      Action|Crime|Thriller  
\end{Verbatim}
            
    \begin{Verbatim}[commandchars=\\\{\}]
{\color{incolor}In [{\color{incolor}14}]:} \PY{c+c1}{\PYZsh{}function that separates the genres for each movie }
         \PY{k}{def} \PY{n+nf}{add\PYZus{}genre}\PY{p}{(}\PY{n}{column}\PY{p}{)}\PY{p}{:}
             \PY{c+c1}{\PYZsh{}will take a column, and separate the string from the \PYZsq{}|\PYZsq{}}
             \PY{n}{data} \PY{o}{=} \PY{n}{clean\PYZus{}mdata}\PY{p}{[}\PY{n}{column}\PY{p}{]}\PY{o}{.}\PY{n}{str}\PY{o}{.}\PY{n}{cat}\PY{p}{(}\PY{n}{sep} \PY{o}{=} \PY{l+s+s1}{\PYZsq{}}\PY{l+s+s1}{|}\PY{l+s+s1}{\PYZsq{}}\PY{p}{)}
             
             \PY{c+c1}{\PYZsh{}giving pandas series and storing the values separately}
             \PY{n}{data} \PY{o}{=} \PY{n}{pd}\PY{o}{.}\PY{n}{Series}\PY{p}{(}\PY{n}{data}\PY{o}{.}\PY{n}{split}\PY{p}{(}\PY{l+s+s1}{\PYZsq{}}\PY{l+s+s1}{|}\PY{l+s+s1}{\PYZsq{}}\PY{p}{)}\PY{p}{)}
             
             \PY{c+c1}{\PYZsh{}arranging in descending order}
             \PY{n}{count} \PY{o}{=} \PY{n}{data}\PY{o}{.}\PY{n}{value\PYZus{}counts}\PY{p}{(}\PY{n}{ascending} \PY{o}{=} \PY{k+kc}{False}\PY{p}{)}
             
             \PY{k}{return} \PY{n}{count}
\end{Verbatim}


    \begin{Verbatim}[commandchars=\\\{\}]
{\color{incolor}In [{\color{incolor}15}]:} \PY{c+c1}{\PYZsh{}variable to store the retured value}
         \PY{n}{genre\PYZus{}count} \PY{o}{=} \PY{n}{add\PYZus{}genre}\PY{p}{(}\PY{l+s+s1}{\PYZsq{}}\PY{l+s+s1}{genres}\PY{l+s+s1}{\PYZsq{}}\PY{p}{)}
\end{Verbatim}


    \begin{Verbatim}[commandchars=\\\{\}]
{\color{incolor}In [{\color{incolor}16}]:} \PY{c+c1}{\PYZsh{} Formula for gross profit: Revenue \PYZhy{} Budget}
         \PY{n}{clean\PYZus{}mdata}\PY{p}{[}\PY{l+s+s1}{\PYZsq{}}\PY{l+s+s1}{gross\PYZus{}profit}\PY{l+s+s1}{\PYZsq{}}\PY{p}{]} \PY{o}{=} \PY{n}{clean\PYZus{}mdata}\PY{p}{[}\PY{l+s+s1}{\PYZsq{}}\PY{l+s+s1}{revenue}\PY{l+s+s1}{\PYZsq{}}\PY{p}{]} \PY{o}{\PYZhy{}} \PY{n}{clean\PYZus{}mdata}\PY{p}{[}\PY{l+s+s1}{\PYZsq{}}\PY{l+s+s1}{budget}\PY{l+s+s1}{\PYZsq{}}\PY{p}{]}
\end{Verbatim}


    \begin{Verbatim}[commandchars=\\\{\}]
{\color{incolor}In [{\color{incolor}17}]:} \PY{n}{clean\PYZus{}mdata}\PY{o}{.}\PY{n}{head}\PY{p}{(}\PY{p}{)}
\end{Verbatim}


\begin{Verbatim}[commandchars=\\\{\}]
{\color{outcolor}Out[{\color{outcolor}17}]:}    popularity     budget     revenue  runtime  \textbackslash{}
         0      32.986  150000000  1513528810      124   
         1      28.420  150000000   378436354      120   
         2      13.113  110000000   295238201      119   
         3      11.173  200000000  2068178225      136   
         4       9.335  190000000  1506249360      137   
         
                                               genres  gross\_profit  
         0  Action|Adventure|Science Fiction|Thriller    1363528810  
         1  Action|Adventure|Science Fiction|Thriller     228436354  
         2         Adventure|Science Fiction|Thriller     185238201  
         3   Action|Adventure|Science Fiction|Fantasy    1868178225  
         4                      Action|Crime|Thriller    1316249360  
\end{Verbatim}
            
    \begin{Verbatim}[commandchars=\\\{\}]
{\color{incolor}In [{\color{incolor}18}]:} \PY{c+c1}{\PYZsh{}Defining these variables here to see a statisitcal correlation in }
         \PY{c+c1}{\PYZsh{}bivariate analysis. Separating my independet and dependent variables}
         \PY{n}{pop} \PY{o}{=} \PY{n}{clean\PYZus{}mdata}\PY{p}{[}\PY{l+s+s1}{\PYZsq{}}\PY{l+s+s1}{popularity}\PY{l+s+s1}{\PYZsq{}}\PY{p}{]}
         \PY{n}{ygp} \PY{o}{=} \PY{n}{clean\PYZus{}mdata}\PY{p}{[}\PY{l+s+s1}{\PYZsq{}}\PY{l+s+s1}{gross\PYZus{}profit}\PY{l+s+s1}{\PYZsq{}}\PY{p}{]}
         
         \PY{n}{xgen} \PY{o}{=} \PY{n}{clean\PYZus{}mdata}\PY{p}{[}\PY{l+s+s1}{\PYZsq{}}\PY{l+s+s1}{genres}\PY{l+s+s1}{\PYZsq{}}\PY{p}{]}
         \PY{n}{xrev} \PY{o}{=} \PY{n}{clean\PYZus{}mdata}\PY{p}{[}\PY{l+s+s1}{\PYZsq{}}\PY{l+s+s1}{revenue}\PY{l+s+s1}{\PYZsq{}}\PY{p}{]}
         \PY{n}{xbudget} \PY{o}{=} \PY{n}{clean\PYZus{}mdata}\PY{p}{[}\PY{l+s+s1}{\PYZsq{}}\PY{l+s+s1}{budget}\PY{l+s+s1}{\PYZsq{}}\PY{p}{]}
\end{Verbatim}


    \subsection{Exploratory Data Analysis}\label{exploratory-data-analysis}

\subsubsection{Research Question 1 (What is the relationship between
popularity of a movie and it's
budget?)}\label{research-question-1-what-is-the-relationship-between-popularity-of-a-movie-and-its-budget}

    Below is a scatter plot that decribes the relationship between budget
and popularity. Budget is the independent variable and popularity is the
dependent variable. The graph to the left shows the relationship in log,
while the plot to the right shows the relationship on the varaibles
regular scale. The log transformation has spread out the data a bit, in
order to see how much budget influences popularity. In most cases, the
higher the budget the more popular a movie was. In a few cases, we can
see that a low budget had a middle popularity. The same goes for a few
high budget movies that had a very low popularity. So, overall, if the
movie had a high budget it looks to correlate with moderate to high
popularity. Popularity and budget are also run through a Pearson's R
correlation and are statistically significant.

    \begin{Verbatim}[commandchars=\\\{\}]
{\color{incolor}In [{\color{incolor}19}]:} \PY{n}{fig}\PY{p}{,} \PY{p}{(}\PY{n}{ax1}\PY{p}{,}\PY{n}{ax2}\PY{p}{)} \PY{o}{=} \PY{n}{plt}\PY{o}{.}\PY{n}{subplots}\PY{p}{(}\PY{l+m+mi}{1}\PY{p}{,} \PY{l+m+mi}{2}\PY{p}{,} \PY{n}{figsize}\PY{o}{=}\PY{p}{(}\PY{l+m+mi}{8}\PY{p}{,}\PY{l+m+mi}{4}\PY{p}{)}\PY{p}{)}
         \PY{n}{ax1}\PY{o}{.}\PY{n}{plot}\PY{p}{(}\PY{n}{clean\PYZus{}mdata}\PY{p}{[}\PY{l+s+s1}{\PYZsq{}}\PY{l+s+s1}{budget}\PY{l+s+s1}{\PYZsq{}}\PY{p}{]}\PY{p}{,} \PY{n}{clean\PYZus{}mdata}\PY{p}{[}\PY{l+s+s1}{\PYZsq{}}\PY{l+s+s1}{popularity}\PY{l+s+s1}{\PYZsq{}}\PY{p}{]}\PY{p}{,} \PY{l+s+s1}{\PYZsq{}}\PY{l+s+s1}{.}\PY{l+s+s1}{\PYZsq{}}\PY{p}{)}
         \PY{n}{ax1}\PY{o}{.}\PY{n}{set\PYZus{}yscale}\PY{p}{(}\PY{l+s+s1}{\PYZsq{}}\PY{l+s+s1}{log}\PY{l+s+s1}{\PYZsq{}}\PY{p}{)}
         \PY{n}{ax1}\PY{o}{.}\PY{n}{set\PYZus{}xscale}\PY{p}{(}\PY{l+s+s1}{\PYZsq{}}\PY{l+s+s1}{log}\PY{l+s+s1}{\PYZsq{}}\PY{p}{)}
         \PY{n}{ax1}\PY{o}{.}\PY{n}{set\PYZus{}xlabel}\PY{p}{(}\PY{l+s+s1}{\PYZsq{}}\PY{l+s+s1}{budget}\PY{l+s+s1}{\PYZsq{}}\PY{p}{)}
         \PY{n}{ax1}\PY{o}{.}\PY{n}{set\PYZus{}ylabel}\PY{p}{(}\PY{l+s+s1}{\PYZsq{}}\PY{l+s+s1}{popularity}\PY{l+s+s1}{\PYZsq{}}\PY{p}{)}
         \PY{n}{ax2}\PY{o}{.}\PY{n}{plot}\PY{p}{(}\PY{n}{clean\PYZus{}mdata}\PY{p}{[}\PY{l+s+s1}{\PYZsq{}}\PY{l+s+s1}{budget}\PY{l+s+s1}{\PYZsq{}}\PY{p}{]}\PY{p}{,} \PY{n}{clean\PYZus{}mdata}\PY{p}{[}\PY{l+s+s1}{\PYZsq{}}\PY{l+s+s1}{popularity}\PY{l+s+s1}{\PYZsq{}}\PY{p}{]}\PY{p}{,} \PY{l+s+s1}{\PYZsq{}}\PY{l+s+s1}{.}\PY{l+s+s1}{\PYZsq{}}\PY{p}{)}
         \PY{n}{ax2}\PY{o}{.}\PY{n}{set\PYZus{}xlabel}\PY{p}{(}\PY{l+s+s1}{\PYZsq{}}\PY{l+s+s1}{budget}\PY{l+s+s1}{\PYZsq{}}\PY{p}{)}
         \PY{n}{ax2}\PY{o}{.}\PY{n}{set\PYZus{}ylabel}\PY{p}{(}\PY{l+s+s1}{\PYZsq{}}\PY{l+s+s1}{popularity}\PY{l+s+s1}{\PYZsq{}}\PY{p}{)}
\end{Verbatim}


\begin{Verbatim}[commandchars=\\\{\}]
{\color{outcolor}Out[{\color{outcolor}19}]:} Text(0,0.5,'popularity')
\end{Verbatim}
            
    \begin{center}
    \adjustimage{max size={0.9\linewidth}{0.9\paperheight}}{output_25_1.png}
    \end{center}
    { \hspace*{\fill} \\}
    
    \begin{Verbatim}[commandchars=\\\{\}]
{\color{incolor}In [{\color{incolor}20}]:} \PY{c+c1}{\PYZsh{}Budget and Population are statistically significant with an alpha of 0.05}
         \PY{n}{pearsonr}\PY{p}{(}\PY{n}{xbudget}\PY{p}{,} \PY{n}{pop}\PY{p}{)}
\end{Verbatim}


\begin{Verbatim}[commandchars=\\\{\}]
{\color{outcolor}Out[{\color{outcolor}20}]:} (0.4469866930400051, 1.0513638511750907e-188)
\end{Verbatim}
            
    \subsubsection{Research Question 2 (Is there an association between a
movies' revenue and how popular it
is?)}\label{research-question-2-is-there-an-association-between-a-movies-revenue-and-how-popular-it-is}

    Below is a scatter plot between the variables popularity and revenue.
THe plot on the left is in log scale and the plot of the right is the
default values. The plot that is in log shows that the higher the
revenue the more popular the movie was. There were cases that show
movies with high revenue were not very popular, but the majority shows
that if a movie had a medium revenue then it was popular. These
varaibles are statisitcally significant, which we can tell by running
them with a Pearson's R correlation test.

    \begin{Verbatim}[commandchars=\\\{\}]
{\color{incolor}In [{\color{incolor}21}]:} \PY{c+c1}{\PYZsh{}clean\PYZus{}mdata.plot(x=\PYZsq{}revenue\PYZsq{},y=\PYZsq{}popularity\PYZsq{},kind=\PYZsq{}scatter\PYZsq{})}
         \PY{n}{fig}\PY{p}{,} \PY{p}{(}\PY{n}{ax1}\PY{p}{,} \PY{n}{ax2}\PY{p}{)} \PY{o}{=} \PY{n}{plt}\PY{o}{.}\PY{n}{subplots}\PY{p}{(}\PY{l+m+mi}{1}\PY{p}{,} \PY{l+m+mi}{2}\PY{p}{,} \PY{n}{figsize}\PY{o}{=}\PY{p}{(}\PY{l+m+mi}{8}\PY{p}{,}\PY{l+m+mi}{4}\PY{p}{)}\PY{p}{)}
         \PY{n}{ax1}\PY{o}{.}\PY{n}{plot}\PY{p}{(}\PY{n}{clean\PYZus{}mdata}\PY{p}{[}\PY{l+s+s1}{\PYZsq{}}\PY{l+s+s1}{revenue}\PY{l+s+s1}{\PYZsq{}}\PY{p}{]}\PY{p}{,} \PY{n}{clean\PYZus{}mdata}\PY{p}{[}\PY{l+s+s1}{\PYZsq{}}\PY{l+s+s1}{popularity}\PY{l+s+s1}{\PYZsq{}}\PY{p}{]}\PY{p}{,} \PY{l+s+s1}{\PYZsq{}}\PY{l+s+s1}{.}\PY{l+s+s1}{\PYZsq{}}\PY{p}{)}
         \PY{n}{ax1}\PY{o}{.}\PY{n}{set\PYZus{}yscale}\PY{p}{(}\PY{l+s+s1}{\PYZsq{}}\PY{l+s+s1}{log}\PY{l+s+s1}{\PYZsq{}}\PY{p}{)}
         \PY{n}{ax1}\PY{o}{.}\PY{n}{set\PYZus{}xscale}\PY{p}{(}\PY{l+s+s1}{\PYZsq{}}\PY{l+s+s1}{log}\PY{l+s+s1}{\PYZsq{}}\PY{p}{)}
         \PY{n}{ax1}\PY{o}{.}\PY{n}{set\PYZus{}xlabel}\PY{p}{(}\PY{l+s+s1}{\PYZsq{}}\PY{l+s+s1}{revenue}\PY{l+s+s1}{\PYZsq{}}\PY{p}{)}
         \PY{n}{ax1}\PY{o}{.}\PY{n}{set\PYZus{}ylabel}\PY{p}{(}\PY{l+s+s1}{\PYZsq{}}\PY{l+s+s1}{popularity}\PY{l+s+s1}{\PYZsq{}}\PY{p}{)}
         \PY{n}{ax2}\PY{o}{.}\PY{n}{plot}\PY{p}{(}\PY{n}{clean\PYZus{}mdata}\PY{p}{[}\PY{l+s+s1}{\PYZsq{}}\PY{l+s+s1}{revenue}\PY{l+s+s1}{\PYZsq{}}\PY{p}{]}\PY{p}{,} \PY{n}{clean\PYZus{}mdata}\PY{p}{[}\PY{l+s+s1}{\PYZsq{}}\PY{l+s+s1}{popularity}\PY{l+s+s1}{\PYZsq{}}\PY{p}{]}\PY{p}{,} \PY{l+s+s1}{\PYZsq{}}\PY{l+s+s1}{.}\PY{l+s+s1}{\PYZsq{}}\PY{p}{)}
         \PY{n}{ax2}\PY{o}{.}\PY{n}{set\PYZus{}xlabel}\PY{p}{(}\PY{l+s+s1}{\PYZsq{}}\PY{l+s+s1}{revenue}\PY{l+s+s1}{\PYZsq{}}\PY{p}{)}
         \PY{n}{ax2}\PY{o}{.}\PY{n}{set\PYZus{}ylabel}\PY{p}{(}\PY{l+s+s1}{\PYZsq{}}\PY{l+s+s1}{popularity}\PY{l+s+s1}{\PYZsq{}}\PY{p}{)}
\end{Verbatim}


\begin{Verbatim}[commandchars=\\\{\}]
{\color{outcolor}Out[{\color{outcolor}21}]:} Text(0,0.5,'popularity')
\end{Verbatim}
            
    \begin{center}
    \adjustimage{max size={0.9\linewidth}{0.9\paperheight}}{output_29_1.png}
    \end{center}
    { \hspace*{\fill} \\}
    
    \begin{Verbatim}[commandchars=\\\{\}]
{\color{incolor}In [{\color{incolor}22}]:} \PY{n}{pearsonr}\PY{p}{(}\PY{n}{xrev}\PY{p}{,} \PY{n}{pop}\PY{p}{)}
\end{Verbatim}


\begin{Verbatim}[commandchars=\\\{\}]
{\color{outcolor}Out[{\color{outcolor}22}]:} (0.6155346545546607, 0.0)
\end{Verbatim}
            
    \subsubsection{Research Question 3 (Is there a relationship between
genre and gross
profit?)}\label{research-question-3-is-there-a-relationship-between-genre-and-gross-profit}

    Below is a histogram of genre and gross profit. Which genre's did had
the highest gross profit? Action, Adventure, Science Fiction,Family,
Fantasy, and Animation. The gren's that did the worst are: Foreign,
Documentary, Horror, and TV Movie.

    \begin{Verbatim}[commandchars=\\\{\}]
{\color{incolor}In [{\color{incolor}23}]:} \PY{n}{genre\PYZus{}profit} \PY{o}{=} \PY{n}{genre\PYZus{}count}\PY{o}{.}\PY{n}{copy}\PY{p}{(}\PY{p}{)}
         \PY{k}{for} \PY{n}{genre} \PY{o+ow}{in} \PY{n}{genre\PYZus{}count}\PY{o}{.}\PY{n}{keys}\PY{p}{(}\PY{p}{)}\PY{p}{:}
             \PY{n}{has\PYZus{}genre} \PY{o}{=} \PY{n}{clean\PYZus{}mdata}\PY{p}{[}\PY{l+s+s1}{\PYZsq{}}\PY{l+s+s1}{genres}\PY{l+s+s1}{\PYZsq{}}\PY{p}{]}\PY{o}{.}\PY{n}{str}\PY{o}{.}\PY{n}{contains}\PY{p}{(}\PY{n}{genre}\PY{p}{)}
             \PY{n}{genre\PYZus{}profit}\PY{p}{[}\PY{n}{genre}\PY{p}{]} \PY{o}{=} \PY{n}{clean\PYZus{}mdata}\PY{p}{[}\PY{l+s+s1}{\PYZsq{}}\PY{l+s+s1}{gross\PYZus{}profit}\PY{l+s+s1}{\PYZsq{}}\PY{p}{]}\PY{p}{[}\PY{n}{has\PYZus{}genre}\PY{p}{]}\PY{o}{.}\PY{n}{mean}\PY{p}{(}\PY{p}{)}
         
         \PY{c+c1}{\PYZsh{}ax = genre\PYZus{}profit.plot(kind=\PYZsq{}bar\PYZsq{}, title=\PYZsq{}Avg Gross Profit By Genre\PYZsq{})}
         \PY{c+c1}{\PYZsh{}ax.set\PYZus{}ylabel(\PYZsq{}Avg Gross Profit\PYZsq{})}
         \PY{c+c1}{\PYZsh{}ax.set\PYZus{}xlabel(\PYZsq{}Genres\PYZsq{})}
\end{Verbatim}


    \begin{Verbatim}[commandchars=\\\{\}]
{\color{incolor}In [{\color{incolor}24}]:} \PY{n}{sns}\PY{o}{.}\PY{n}{set}\PY{p}{(}\PY{n}{font\PYZus{}scale}\PY{o}{=}\PY{l+m+mf}{1.1}\PY{p}{)}
         \PY{n}{sns}\PY{o}{.}\PY{n}{set\PYZus{}style}\PY{p}{(}\PY{l+s+s1}{\PYZsq{}}\PY{l+s+s1}{darkgrid}\PY{l+s+s1}{\PYZsq{}}\PY{p}{)}
         \PY{n}{gp} \PY{o}{=} \PY{n}{genre\PYZus{}profit}\PY{o}{.}\PY{n}{to\PYZus{}frame}\PY{p}{(}\PY{l+s+s1}{\PYZsq{}}\PY{l+s+s1}{profit}\PY{l+s+s1}{\PYZsq{}}\PY{p}{)}
         \PY{n}{gp}\PY{o}{.}\PY{n}{reset\PYZus{}index}\PY{p}{(}\PY{n}{level}\PY{o}{=}\PY{l+m+mi}{0}\PY{p}{,} \PY{n}{inplace}\PY{o}{=}\PY{k+kc}{True}\PY{p}{)}
         \PY{n}{gp}\PY{o}{.}\PY{n}{columns} \PY{o}{=} \PY{p}{[}\PY{l+s+s1}{\PYZsq{}}\PY{l+s+s1}{genre}\PY{l+s+s1}{\PYZsq{}}\PY{p}{,} \PY{l+s+s1}{\PYZsq{}}\PY{l+s+s1}{profit}\PY{l+s+s1}{\PYZsq{}}\PY{p}{]}
         \PY{n}{plot} \PY{o}{=} \PY{n}{sns}\PY{o}{.}\PY{n}{catplot}\PY{p}{(}\PY{n}{x}\PY{o}{=}\PY{l+s+s1}{\PYZsq{}}\PY{l+s+s1}{genre}\PY{l+s+s1}{\PYZsq{}}\PY{p}{,} \PY{n}{y}\PY{o}{=}\PY{l+s+s1}{\PYZsq{}}\PY{l+s+s1}{profit}\PY{l+s+s1}{\PYZsq{}}\PY{p}{,} \PY{n}{kind}\PY{o}{=}\PY{l+s+s1}{\PYZsq{}}\PY{l+s+s1}{bar}\PY{l+s+s1}{\PYZsq{}}\PY{p}{,} \PY{n}{data}\PY{o}{=}\PY{n}{gp}\PY{p}{)}
         \PY{n}{plot}\PY{o}{.}\PY{n}{set\PYZus{}xticklabels}\PY{p}{(}\PY{n}{rotation}\PY{o}{=}\PY{l+m+mi}{85}\PY{p}{)}
         \PY{n}{plot}\PY{o}{.}\PY{n}{fig}\PY{o}{.}\PY{n}{get\PYZus{}axes}\PY{p}{(}\PY{p}{)}\PY{p}{[}\PY{l+m+mi}{0}\PY{p}{]}\PY{o}{.}\PY{n}{set\PYZus{}ylabel}\PY{p}{(}\PY{l+s+s1}{\PYZsq{}}\PY{l+s+s1}{Avg Profit per Movie}\PY{l+s+s1}{\PYZsq{}}\PY{p}{)}
         \PY{n}{plot}\PY{o}{.}\PY{n}{fig}\PY{o}{.}\PY{n}{get\PYZus{}axes}\PY{p}{(}\PY{p}{)}\PY{p}{[}\PY{l+m+mi}{0}\PY{p}{]}\PY{o}{.}\PY{n}{set\PYZus{}xlabel}\PY{p}{(}\PY{l+s+s1}{\PYZsq{}}\PY{l+s+s1}{Movie Genres}\PY{l+s+s1}{\PYZsq{}}\PY{p}{)}
\end{Verbatim}


\begin{Verbatim}[commandchars=\\\{\}]
{\color{outcolor}Out[{\color{outcolor}24}]:} Text(0.5,10.256,'Movie Genres')
\end{Verbatim}
            
    \begin{center}
    \adjustimage{max size={0.9\linewidth}{0.9\paperheight}}{output_34_1.png}
    \end{center}
    { \hspace*{\fill} \\}
    
     \#\# Conclusions

Overall, important information can be gathered when looking at the budet
and revenue for the popularity of movies. The above plots, on average,
show that with a decent budget a movie can be popular. There are a few
outliers but for most part, looking at the groupings of each movie's
budget showed that a movie needs a moderate budget to do well. A movie's
revenue is also important to look at. Surprisingly, a small cohert of
movies that did not have a high revenue return was still considered to
be popular. For the most part though, revenue is a good indicator of how
popular a movie is.


    % Add a bibliography block to the postdoc
    
    
    
    \end{document}
